\documentclass[]{mdkrause_cv_openfont}
\usepackage{fancyhdr}
 
\pagestyle{fancy}
\fancyhf{}
 
\begin{document}

%%%%%%%%%%%%%%%%%%%%%%%%%%%%%%%%%%%%%%
%
%     LAST UPDATED DATE
%
%%%%%%%%%%%%%%%%%%%%%%%%%%%%%%%%%%%%%%
%\lastupdated

%%%%%%%%%%%%%%%%%%%%%%%%%%%%%%%%%%%%%%
%
%     TITLE NAME
%
%%%%%%%%%%%%%%%%%%%%%%%%%%%%%%%%%%%%%%
\namesection{Matheus D.}{Krause}{ 
\href{mailto:krause.d.matheus@gmail.com}{krause.d.matheus@gmail.com} | +1 (515) 708-3842 | \href{https://mdkrause.github.io//}{about.me \ExternalLink} 
}

%%%%%%%%%%%%%%%%%%%%%%%%%%%%%%%%%%%%%%
%
%     COLUMN ONE
%
%%%%%%%%%%%%%%%%%%%%%%%%%%%%%%%%%%%%%%

\begin{minipage}[t]{1\textwidth} 

\sectionsep

Plant breeding is a fascinating and important science to ensure high quality and sufficient quantities of food are available for our growing population. This endeavor requires motivated scientists. In this scenario, my research interests comprise biometry, biostatistics, quantitative and statistical genetics. I am also excited about the future of agriculture: I believe we must develop approaches to mitigate harmful environmental impacts and improve efficiency at all levels of the value chain.

\sectionsep

%%%%%%%%%%%%%%%%%%%%%%%%%%%%%%%%%%%%%%
%     EDUCATION
%%%%%%%%%%%%%%%%%%%%%%%%%%%%%%%%%%%%%%

\section{Education} 

\sectionsep

\subsection{Ph.D. Candidate in Plant Breeding with Statistics minor (Exp. Spring 2023)}
\descript{Iowa State University - Ames, Iowa, USA | Major Professor: Dr. William D. Beavis}
\project{Projects:} Unraveling genotype by environment variation to identify mega-environments with genetic and non-genetic factors. \\ 
\hspace{1.55 cm} Development of estimation methods to obtain unbiased realized rate of genetic gain using routine field trials.

\sectionsep
\vspace{0.2 mm}

\subsection{M.S in Plant Breeding and Genetics (2018)}
\descript{University of São Paulo - Piracicaba, SP, Brazil | Major Professor: Dr. Antonio A. F. Garcia}
\project{Title:} Boosting predictive ability of maize hybrids via genotype by environment interaction under multivariate GBLUP models.

\sectionsep
\vspace{0.2 mm}

\subsection{B.A in Agronomy (2015)}
\descript{Londrina State University - Londrina, PR, Brazil | Major Professor: Dr. Josué M. Ferreira}
\project{Title:} Combining ability of maize inbred lines S$_9$ derived from ST15 and S709 synthetics.

\sectionsep
\vspace{0.2 mm}

\subsection{Exchange Student in Plant Breeding \& Engineering for Mediterranean and Tropical Areas (2013/14)}
\descript{Montpellier SupAgro - Montpellier, France | Under Drs. Ricardo Ralisch and Jean-Luc Regnard}
\project{Title:} Connaissance des Géniteurs Maïs (Confidential to Limagrain Europe and Montpellier SupAGro).

\sectionsep

%%%%%%%%%%%%%%%%%%%%%%%%%%%%%%%%%%%%%%
%     EXPERIENCE
%%%%%%%%%%%%%%%%%%%%%%%%%%%%%%%%%%%%%%

\section{Research Experience}

\sectionsep

\runsubsection{Bill Beavis's G.F. Sprague Quantitative Genetics Group}
\descript{| Graduate research assistant}
\project{May 2018 - Present | Department of Agronomy, Iowa State University} \\
%\vspace{\topsep} 
- Develop a metric to estimate realized genetic gains on an annual basis using soybean routine field trials. \\
- Unraveling genotype by environment variation to identify mega-environments with genetic and non-genetic factors. \\
- Develop software to correct partially informative markers from backcross one derived (BC1) double-haploid (DH) lines. \\
- Genome-wise association for complex binary (case-control) and continuous traits. \\
- Consult several graduate students (+20) with plant breeding data analysis and \texttt{R} programming. 

\sectionsep
\vspace{0.2 mm}

\runsubsection{Statistical Genetics Lab}
\descript{| Graduate research assistant}
\project{March 2016 - March 2018 | Department of Genetics, University of São Paulo} \\
%\vspace{\topsep} 
- Apply multivariate linear mixed models to predict yield performance of single-cross maize hybrids based on information from genomics and genotype by environment interaction. 

\sectionsep
\vspace{0.2 mm}

\runsubsection{Tropical Breeding \& Genetics}
\descript{| Intern}
\project{Aug 2015 - Nov 2015 | Soybean Breeding} \\
%\vspace{\topsep}
- Plant phenotyping for soybean cyst nematode, \textit{Phytophthora sojae} and \textit{Sclerotinia sclerotiorum}, generation advancement in greenhouses, crossing blocks, seed packing, and general activities in the biotechnology laboratory.

\sectionsep
\vspace{0.2 mm}

\runsubsection{Laboratory of Maize Breeding}
\descript{| Undergraduate research assistant}
\project{Jun 2010 - Jun 2015 | Department of Biology, Londrina State University} \\
%\vspace{\topsep} 
- Development and evaluation of inbred lines for common and sweet corn via classical and DH techniques. \\
- Evaluation and development of single-cross hybrids and synthetic populations for both common and sweet corn. \\
- Participation in an extension project to teach farmers how to produce their seeds of synthetic populations.

\sectionsep
\vspace{0.2 mm}

\runsubsection{Limagrain Europe}
\descript{| Intern}
\project{March 2014 - Jun 2014 | Department of Seed Production to Research} \\
%\vspace{\topsep}
- Data mining to support early decision-making on the profitability of female maize inbred lines (seed production). \\
- Stability analysis to identify stable check cultivars to be used in seed production. 

\end{minipage} 
\hfill

\begin{minipage}[t]{1\textwidth} 

%%%%%%%%%%%%%%%%%%%%%%%%%%%%%%%%%%%%%%
%     PUBLICATIONS
%%%%%%%%%%%%%%%%%%%%%%%%%%%%%%%%%%%%%%

\section{Journal Publications} 

\sectionsep
\textbf{Krause, MD}; Dias, KOG; Singh, AK; Beavis, WD. Using soybean historical field trial data to study genotype by environment variation and identify mega-environments with the integration of genetic and non-genetic factors. Submitted to Field Crops Research, 2022. \href{https://www.biorxiv.org/content/10.1101/2022.04.11.487885v3.abstract}{DOI: 10.1101/2022.04.11.487885}. \ExternalLink

\sectionsep
Dias, KOG; dos Santos, JPR; \textbf{Krause, MD}; Piepho, HP; Guimarães, LJM; Pastina, MM; Garcia, AAF. Leveraging probability concepts for cultivar recommendation in multi-environment trials. Theoretical and Applied Genetics, 2022. \href{https://link.springer.com/article/10.1007/s00122-022-04041-y}{DOI: 10.1007/s00122-022-04041-y}. \ExternalLink 

\sectionsep
Montes, CR; Fox, C; Sanz-Sáez, A; Serbin, SP; Kumagai, E; \textbf{Krause, MD}; Xavier, A; Specht, J; Beavis, WD; Bernacchi, CJ; Diers, BW; Ainsworth, EA. High-throughput characterization, correlation, and mapping of leaf photosynthetic and functional traits in the soybean (\textit{Glycine max}) nested association mapping population. Genetics, 2022. \href{https://academic.oup.com/genetics/advance-article/doi/10.1093/genetics/iyac065/6572345?login=true}{DOI: 10.1093/genetics/iyac065}. \ExternalLink

\sectionsep
Verzegnazzi, AL; Santos, I; \textbf{Krause, MD}; Hufford, M; Frei, UK; Campbell, J; Almeida, VC; Zuffo, LT; Boerman, N; Lübberstedt, T. Major locus for spontaneous haploid genome doubling detected by a case–control GWAS in exotic maize germplasm. Theoretical and Applied Genetics, 2021. \href{https://link.springer.com/article/10.1007/s00122-021-03780-8}{DOI: 10.1002/csc2.20253}. \ExternalLink 

\sectionsep
\textbf{Krause, MD}; Dias, KOG; dos Santos, JPR; Oliveira, AAO; Guimarães, LJM; Pastina, MM; Margarido, GRA; Garcia, AAF. Boosting predictive ability of tropical maize hybrids via genotype by environment interaction under multivariate GBLUP models. Crop Science, 2020. \href{https://acsess.onlinelibrary.wiley.com/doi/full/10.1002/csc2.20253}{DOI: 10.1002/csc2.20253}. \ExternalLink 

\sectionsep
Sekiya, A; Pestana, JK; da Silva, MGB; \textbf{Krause, MD}; da Silva, CRM; Ferreira, JM. Tropical supersweet corn haploid induction and ploidy determination at seedling stages. Brazilian Journal of Agricultural Research, 2020. \href{https://www.scielo.br/pdf/pab/v55/1678-3921-pab-55-e00968.pdf}{DOI: 10.1002/csc2.20253}. \ExternalLink 

\sectionsep
Xavier, LFS; Pestana, JK; Sekiya, A; \textbf{Krause, MD}; Moreira, RMP; Ferreira, JM. Partial diallel and potential of super sweet corn inbred lines bt$_2$ to obtain hybrids. Horticultura Brasileira, 2019. \href{http://www.scielo.br/scielo.php?script=sci_arttext&pid=S0102-05362019000300278}{DOI: 10.1590/s0102-053620190305}. \ExternalLink 

\sectionsep
Koltun, A; Cavalcante, AP; Lopes, KBA; \textbf{Krause, MD}; Marino, TP; Oliveira, ALM; Ferreira JM. Performance of maize hybrids from a partial diallel in association with \textit{Azospirillum}. African Journal of Agricultural Research, 2018. \href{https://academicjournals.org/journal/AJAR/article-abstract/B4A2A1B57541}{DOI: 10.5897/ajar2018.13077}. \ExternalLink

\sectionsep
\project{+ 25 Abstracts and 40 technical publications in agricultural newspapers.}

\sectionsep

%%%%%%%%%%%%%%%%%%%%%%%%%%%%%%%%%%%%%%
%     SKILLS
%%%%%%%%%%%%%%%%%%%%%%%%%%%%%%%%%%%%%%

\section{Additional information}

\sectionsep

\descript{Teaching \& Seminar}

\begin{tightemize}
	\vspace{3 mm}
	\item \project{2020/22} Lecture on \emph{AlphaSimR} in Agron 523 (Molecular P. Breed., graduate-level course, Dr. Lübberstedt). 
	\item \project{2022/23} Invited to speak in the ``Big Data: Manage your data before your data kills you" session at the Plant and Animal Genome Conference Plant and Animal Genome Conference 2022/2023.
	\item \project{2017} Instructor of \texttt{R} programming course at the University of São Paulo.
\end{tightemize}

\sectionsep

\descript{Programming Skills}
Advanced \texttt{R} programming language, basic Python and Julia, Linux, \LaTeX \hspace{1 mm} \textbullet{} Developer of the \texttt{R} package \href{https://cran.r-project.org/web/packages/SoyURT/index.html}{SoyURT} \ExternalLink

\sectionsep

\descript{Language}
Portuguese (native), English, basic French

\sectionsep

\descript{Leadership}
\begin{tightemize}
	\item \project{2020-2021} Committee member in the 7th and 8th annual R. F. Baker Plant Breeding Symposium at Iowa State University.
	\item \project{2011-2013} Student representative of Agronomy in the Regional Council of Engineering and Agronomy.
\end{tightemize}

\sectionsep

\descript{Awards}
C. R. Weber for Excellence in Plant Breeding (Spring 2022) \textbullet{} Bayer Travel (2020) \& Mentoring Program (2022). \\

\sectionsep

\descript{Matheus enjoys...}
Spending time with his family and friends \textbullet{} Reading \textbullet{} Fishing \textbullet{} Being outdoors \textbullet{} Playing guitar \textbullet{} Riding motorcycle. \\

\sectionsep
\descript{For more information, please access}

\sectionsep
\sectionsep

% href with includegraphics works with lualatex

\hspace{6 cm} \href{https://www.linkedin.com/in/matheus-dalsente-krause-3012b158/}{\includegraphics[scale=0.125]{./logos/lin.jpeg}} \hspace{2 cm} \href{https://about.me/matheusdalsentekrause}{\includegraphics[scale=0.025]{./logos/g.png}} \hspace{2 cm} \href{https://mdkrause.github.io//}{\includegraphics[scale=0.06]{./logos/github.png}}

\end{minipage} 
\end{document}  \documentclass[]{article}
