\documentclass[]{mdkrause_cv_openfont}
\usepackage{fancyhdr}
 
\pagestyle{fancy}
\fancyhf{}
 
\begin{document}

%%%%%%%%%%%%%%%%%%%%%%%%%%%%%%%%%%%%%%
%
%     LAST UPDATED DATE
%
%%%%%%%%%%%%%%%%%%%%%%%%%%%%%%%%%%%%%%
%\lastupdated

%%%%%%%%%%%%%%%%%%%%%%%%%%%%%%%%%%%%%%
%
%     TITLE NAME
%
%%%%%%%%%%%%%%%%%%%%%%%%%%%%%%%%%%%%%%
\namesection{Matheus D.}{Krause}{ 
\href{mailto:krause.d.matheus@gmail.com}{krause.d.matheus@gmail.com} | +1 (515) 708-3842 | \href{https://mdkrause.github.io//}{about.me \ExternalLink} 
}

%%%%%%%%%%%%%%%%%%%%%%%%%%%%%%%%%%%%%%
%
%     COLUMN ONE
%
%%%%%%%%%%%%%%%%%%%%%%%%%%%%%%%%%%%%%%

\begin{minipage}[t]{1\textwidth} 

%\sectionsep

\vspace{1 mm}

I am a motivated Ph.D. Candidate in Plant Breeding with a Statistics minor at Iowa State University. My field of expertise and research interest include quantitative and statistical genetics, biometry, and biostatistics. I have extensive experience in \texttt{R} programming and contributed to many collaborative projects focused on data analysis. In addition, during my academic career, I had hands-on experience in daily breeding operations in different countries. My expected graduation date is the Spring of 2023.

%\sectionsep

%%%%%%%%%%%%%%%%%%%%%%%%%%%%%%%%%%%%%%
%     EDUCATION
%%%%%%%%%%%%%%%%%%%%%%%%%%%%%%%%%%%%%%

\section{Education} 

\sectionsep

\subsection{Ph.D. Candidate in Plant Breeding with Statistics minor (2023)}
\descript{Iowa State University - Ames, Iowa, USA | Major Professor: Dr. William D. Beavis}
\project{Title:} Untangling genetic and non-genetic factors for soybean breeding optimization.
\sectionsep
%\vspace{0.2 mm}

\subsection{Master's Degree in Plant Breeding and Genetics (2018)}
\descript{University of São Paulo - Piracicaba, SP, Brazil | Major Professor: Dr. Antonio A. F. Garcia}
\project{Title:} Boosting predictive ability of maize hybrids via genotype by environment interaction under multivariate GBLUP models.

\sectionsep
%\vspace{0.2 mm}

\subsection{Bachelor's Degree in Agronomy (2015)}
\descript{Londrina State University - Londrina, PR, Brazil | Major Professor: Dr. Josué M. Ferreira}
\project{Title:} Combining ability of maize inbred lines S$_9$ derived from ST15 and S709 synthetics.

\sectionsep
%\vspace{0.2 mm}

\subsection{Exchange Student in Plant Breeding \& Engineering for Mediterranean and Tropical Areas (2014)}
\descript{SupAgro - Montpellier, France | Supervisors: Drs. Ricardo Ralisch and Jean-Luc Regnard}
\project{Title:} Productibilité semencière de nouveaux géniteurs maïs : faisabilité de l’utilisation de données de production.

\sectionsep

%%%%%%%%%%%%%%%%%%%%%%%%%%%%%%%%%%%%%%
%     EXPERIENCE
%%%%%%%%%%%%%%%%%%%%%%%%%%%%%%%%%%%%%%

\section{Research Experience}

\sectionsep

\runsubsection{Graduate Research Assistant}
\descript{| 2018 - Present, Department of Agronomy, Iowa State University}
%\vspace{\topsep} 
- Development of a metric to estimate realized rate of genetic gain on an annual basis using soybean routine field trials. \\
- Unraveling genotype by environment variation to identify mega-environments with genetic and non-genetic factors. \\
- Development of software to correct partially informative markers from backcross one derived double-haploid lines. \\
- Genome-wide association for complex binary (case-control) and continuous traits. \\
- Mentored several graduate students (20+) with plant breeding data analysis and \texttt{R} programming. 

\sectionsep
%\vspace{0.2 mm}

\runsubsection{Graduate Research Assistant}
\descript{| 2016 - 2017, Department of Genetics, University of São Paulo}
%\vspace{\topsep} 
- Applying multivariate linear mixed models to predict yield performance of single-cross maize hybrids based on information from genomics and genotype by environment interaction. 

\sectionsep
%\vspace{0.2 mm}

\runsubsection{Intern}
\descript{| 2015, Tropical Breeding \& Genetics, Soybean Breeding}
%\project{2015 | Soybean Breeding} \\
- Plant phenotyping for soybean cyst nematode, \textit{Phytophthora sojae} and \textit{Sclerotinia sclerotiorum}, generation advancement in greenhouses, crossing blocks, seed packing, and general activities in the biotechnology laboratory.

\sectionsep
%\vspace{0.2 mm}

\runsubsection{Undergraduate Research Assistant}
\descript{| 2010 - 2015, Department of Biology, Londrina State University}
- Development and evaluation of inbred lines for common and sweet corn via classical and double-haploid techniques. \\
- Evaluation and development of single-cross hybrids and synthetic populations for both common and sweet corn. \\
- Participation in an extension project to teach farmers how to produce their seeds of synthetic populations.

\sectionsep
%\vspace{0.2 mm}

\runsubsection{Intern}
\descript{| 2014, Limagrain Europe, Department of Seed Production to Research}
%\project{2014 | Department of Seed Production to Research} \\
%\vspace{\topsep}
- Data mining to support early decision-making on the profitability of female maize inbred lines (seed production). \\
- Stability analysis to identify stable check cultivars to be used in seed production. 

\sectionsep

\section{Selected Journal Publications} 

\sectionsep

\normalsize

\textbf{Krause, MD}; Dias, KOG; Singh, AK; Beavis, WD. Using soybean historical field trial data to study genotype by environment variation and identify mega-environments with the integration of genetic and non-genetic factors. Submitted to Field Crops Research, 2022. \href{https://www.biorxiv.org/content/10.1101/2022.04.11.487885v3.abstract}{DOI: 10.1101/2022.04.11.487885}. \ExternalLink

\end{minipage} 
\hfill

\begin{minipage}[t]{1\textwidth} 

%%%%%%%%%%%%%%%%%%%%%%%%%%%%%%%%%%%%%%
%     PUBLICATIONS
%%%%%%%%%%%%%%%%%%%%%%%%%%%%%%%%%%%%%%

Dias, KOG; dos Santos, JPR; \textbf{Krause, MD}; Piepho, HP; Guimarães, LJM; Pastina, MM; Garcia, AAF. Leveraging probability concepts for cultivar recommendation in multi-environment trials. Theoretical and Applied Genetics, 2022. \href{https://link.springer.com/article/10.1007/s00122-022-04041-y}{DOI: 10.1007/s00122-022-04041-y}. \ExternalLink 

\sectionsep
Montes, CR; Fox, C; Sanz-Sáez, A; Serbin, SP; Kumagai, E; \textbf{Krause, MD}; Xavier, A; Specht, J; Beavis, WD; Bernacchi, CJ; Diers, BW; Ainsworth, EA. High-throughput characterization, correlation, and mapping of leaf photosynthetic and functional traits in the soybean (\textit{Glycine max}) nested association mapping population. Genetics, 2022. \href{https://academic.oup.com/genetics/advance-article/doi/10.1093/genetics/iyac065/6572345?login=true}{DOI: 10.1093/genetics/iyac065}. \ExternalLink

\sectionsep
Verzegnazzi, AL; Santos, I; \textbf{Krause, MD}; Hufford, M; Frei, UK; Campbell, J; Almeida, VC; Zuffo, LT; Boerman, N; Lübberstedt, T. Major locus for spontaneous haploid genome doubling detected by a case–control GWAS in exotic maize germplasm. Theoretical and Applied Genetics, 2021. \href{https://link.springer.com/article/10.1007/s00122-021-03780-8}{DOI: 10.1002/csc2.20253}. \ExternalLink 

\sectionsep
\textbf{Krause, MD}; Dias, KOG; dos Santos, JPR; Oliveira, AAO; Guimarães, LJM; Pastina, MM; Margarido, GRA; Garcia, AAF. Boosting predictive ability of tropical maize hybrids via genotype by environment interaction under multivariate GBLUP models. Crop Science, 2020. \href{https://acsess.onlinelibrary.wiley.com/doi/full/10.1002/csc2.20253}{DOI: 10.1002/csc2.20253}. \ExternalLink 

\sectionsep

%%%%%%%%%%%%%%%%%%%%%%%%%%%%%%%%%%%%%%
%     SKILLS
%%%%%%%%%%%%%%%%%%%%%%%%%%%%%%%%%%%%%%

\section{Skill set}

\sectionsep

\descript{Specific skiils}
\begin{tightemize}
	\vspace{0.5 mm}
	\item Data science: Linear and generalized mixed models, bayesian inference, stochastic simulations, probability theory. Data mining, curation, and acquisition for statistical inference. Cloud / high performance computing.
	\item Plant breeding: Genomic prediction, QTL mapping, GWAS, GxE analysis, experimental designs.
	\item I am a constant learner eager to collaborate across diverse groups.
\end{tightemize}

\sectionsep

\descript{Programming Experience}
\begin{tightemize}
	\item Advanced \texttt{R} programming language, basic Python and Julia, Linux, and \LaTeX.
	\item Developer of the \texttt{R} package \href{https://cran.r-project.org/web/packages/SoyURT/index.html}{SoyURT.} \ExternalLink
\end{tightemize}

\sectionsep

\descript{Leadership}
\begin{tightemize}
	\item \project{2020-2021} Committee member in the 7th and 8th annual R. F. Baker Plant Breeding Symposium at Iowa State University.
	\item \project{2011-2013} Student representative of Agronomy in the Regional Council of Engineering and Agronomy in Brazil.
\end{tightemize}

\sectionsep

\descript{Language}
English, Portuguese (native), basic French.

\sectionsep

\section{Teaching, Seminar \& Awards}

\sectionsep

\descript{Teaching \& Seminar}
\begin{tightemize}
	\item \project{2023} Talk on ``From PDF Files to Biological Insights into Soybean Breeding: An Example of How Recovered Historical Yield Data Can be Valuable'', in the ``Big data: manage your data before your data kills you'' workshop at the Plant \& Animal Genome Conference, in San Diego, California - USA.
	\item \project{2020 and 2022} Lecture on ``Simulations in Plant Breeding - An emphasis on \emph{AlphaSimR}'' in Agron 523, Molecular Plant Breeding, graduate-level course under Dr. Thomas Lübberstedt at Iowa State University.
	\item \project{2017} Instructor of \texttt{R} programming at the University of São Paulo.
\end{tightemize}

\sectionsep

\descript{Awards}
\begin{tightemize}
	\item \project{2022} C. R. Weber for Excellence in Plant Breeding from the Department of Agronomy at Iowa State University.
	\item \project{2020 and 2022} Bayer Travel Scholarship \& Mentoring Program for Graduate Students and Postdoctoral Researchers.
\end{tightemize}

\vspace{3 mm}

\section{Additional information}

\sectionsep

\descript{Matheus enjoys...}
Spending time with his family and friends \textbullet{} Reading \textbullet{} Fishing \textbullet{} Being outdoors \textbullet{} Playing guitar \textbullet{} Riding motorcycle. \\

\sectionsep
\descript{For more information, please access}

\sectionsep

% href with includegraphics works with lualatex

\hspace{6 cm} \href{https://www.linkedin.com/in/matheus-dalsente-krause-3012b158/}{\includegraphics[scale=0.125]{./logos/lin.jpeg}} \hspace{2 cm}  \href{https://mdkrause.github.io//}{\includegraphics[scale=0.06]{./logos/github.png}}\hspace{2 cm}
\href{https://about.me/matheusdalsentekrause}{\includegraphics[scale=0.025]{./logos/g.png}} 

\end{minipage} 
\end{document}  \documentclass[]{article}
